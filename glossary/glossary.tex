\chapter{Glossary}

\paragraph{Base language} \Compose* is an extension to other programming languages, the base language is the language
that gets extended. So for \Compose*[Java] the programming language Java is the parent language.

\paragraph{\Compose*[Java]} \Compose*[Java] is the Java port of \Compose*.

\paragraph{\Compose*[DotNET]} \Compose*[Net] is the .Net 1.1 port of \Compose*.

\paragraph{\Compose*{} developer} Somebody who works on the development of the Compose* toolset.
This term is used to avoid the ambiguous term \Compose* programmer.

\paragraph{\Compose*{} user} Somebody who uses \Compose*, this does not exclude the
person from being a Compose* developer. This term is used to avoid the ambiguous term Compose* programmer.

\paragraph{Default constructor} A constructor without any argument.

\paragraph{Functional complete} A set of logical operators is functional complete when it can express any operator from
the set~\cite{vanBenthem91}.
So a XOR (exclusive OR) is not in a functional complete set, because $A~XOR~B~=~(A~AND~!B)~OR~(!A~AND~B)$. 
The smallest functional complete set only contains OR and NOT.

\paragraph{Language independence} Compose* is designed to be used for any language, that gives that Compose* needs to be language independent. 
This language in dependency influences Compose* in two ways, first on the syntax and semantics, and second on the under laying mechanisms. 
True language independence means that a concern can be added to an application to a random parent language and it behaves he same as if you would place it in the same application written in a different parent language. 
For the syntax and semantics this means that we cannot add primitive types to \Compose*, because for instance \emph{int}, \emph{char}, and \emph{double}, can have different ranges.
These limitations can be solved by automating some conversions, for instance we could introduce our own primitive type \emph{i} which is an integer of 32 bits, then we can transform this primitive to the primitive value we want by creating a conversion method in \Compose*. 
However a workaround is not a proper solution, what works in most cases might not work in an other case.
These workaround increase the maintainability of the toolset, and are usually very error prone.
For example the primitive example mentioned earlier, primitives are always bound to a certain specification.
The specification for similar primitives can be different across different platforms and parent languages.
For example the \emph{int} primitive in C is defined to be a signed number that is at least the size of a system word and at least 16bits. 
This means that on 32 bit systems this could be either 16 bits or 32 bits depending on the compiler.
Our suggested primitive \emph{i} is a 32 bit signed number, which does not have a direct mapping to the C \emph{int} type.
This can lead to confusion for the \Compose* users, and depending on the conversion method limit the parent language.
%For instance we go further with our primitive \emph{i}, we find ourselves in the situation that this idea only works if the application stays on one platform and one parent language, if we would import a library then we cannot say for certain to what primitive type we must convert. 
%So if we apply the our own defined primitives we limit the use of the parent language, which is something we do not want.

This gives that there are a few basic comments on language independence which are common for all the parts of the
\Compose* language:

\begin{itemize}[noitemsep]
\item Keywords from parent languages have language specific meanings, that means that we cannot use them
directly in \Compose*{}.
\item Language independent elements are Objects, Methods, Packages, and Annotations. However these
might not be available in every language, for instance C does not know Objects and Annotations are not common for all
languages.
\item Workarounds with conversion mechanisms might work, but this is conceptual not sound or it would introduce unwanted limitations.
\item A lot of primitives do not share identical specifications between various languages. Which makes it difficult to use them in a language independent manner.
%\item If something, for example an integer is 32 bit, in most languages, say Java, C\#, C++, Pascal, then it does not mean that it can be made language independence. There is probably one language around where the integer is not 32 bit.
\end{itemize}

\paragraph{Signature of a class} There are various definitions of the signature of a class, the one used in \Compose* is
that the signature of a class is the collection of signatures of the methods of the class. The signature of a class
is the name, return type, and the list of parameter types. The signature of the class in \autoref{glos::signature}
is \{\emph{Sting getName, void setBirthDate(Date), void setBirthDate(String)} \}.

\begin{lstlisting}[caption={Example class in Java},label=glos::signature,float=h,language=Java,style=listing]
public class Person{
	public String getName(){}
	public void setBirthDate(Date date){}
	public void setBirthDate(String date){}
}
\end{lstlisting}

\paragraph{StarLight} The .Net 2 port of \Compose*.

\paragraph{Wild card} The term used for the `*' token. It \Compose* this generally means to disregard the field value.
