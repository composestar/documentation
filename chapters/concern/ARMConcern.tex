% ARM concern entry
\chapter{Concern} \label{chapter:concern}
Concerns are the distinctive building blocks of a \Compose* application, in addition to the building blocks
of the base language. Conceptually concerns are an extension to classes.
Concerns are declared in files with the .cps extension.

A concern has three different parts: zero or more filter modules, an optional superimposition part, and an optional implementation part. The filter modules are superimposed
on classes by the filter module binding field of the superimposition. The implementation part contains
language dependent code of the concern.

\section*{Syntax}
\begin{lstlisting}[caption = {Concern syntax}, label = lst::ARM:concern:syntax,
style = listing, language = ebnf, float = tpb]
Concern ::= `concern' Identifier [`in' Namespace]
            `{' (FilterModule)* [SuperImposition] [Implementation] `}'
Namespace ::= Identifier (`.' Identifier)*
ConcernBlock ::= `{' (FilterModule)* [SuperImposition] [Implementation] `}'               
\end{lstlisting}
The syntax of a concern is shown in \autoref{lst::ARM:concern:syntax}. A concern name must be unique for the package where it is declared. It often has the same name as the .cps file in which it is declared. It is not possible to place two concerns in one file.
The ordering of the filter modules, superimposition, and implementation is fixed.

\section*{Semantics}
The concern is the main language entity of \Compose*, it consists of one or more filter modules,
an optional superimposition part, and an optional implementation part. How these are combined depends on how you use
a \Compose* concern. There are two different types of usage of a concern. We will look here to the usage that comes from the conceptual idea, which assumes that a
class is a concern. 
This means that every class can be written as a concern, with an implementation part and one or more filter modules. To superimpose these concerns onto other classes, a concern is used with a superimposition part; this concern can be seen as a sort of aspect specification.

\section*{Examples}
\begin{lstlisting}[language={Composestar},style=floatlisting, caption={Abstract example of a concern},label={lst::arm::con:concerntemplate}, floatplacement=tbp]
concern aConcern in aNamespace{
  filtermodule A{
    ...
  }
  
  filtermodule B{
    ...
  }
  
  superimposition{
    ...
  }
  
  implementation{
    ...
  }
}
\end{lstlisting}
A concern heading consists of a concern name%, concern parameters which are optional,
and the package in which the concern is defined. The package is also optional, if none is specified the concern is located in the root
of a project, the usage of packages works the same as in other languages that supports packages or namespaces.

In the concern it is possible to create one or more filter modules, one superimposition block
and an implementation part. A possible concern is demonstrated in \autoref{lst::arm::con:concerntemplate}.

\section*{Legality Rules}
\begin{itemize}[noitemsep]
\item The identifier of a concern must be unique for the namespace where it is declared in;
\item The ordering of filter modules, superimposition, and implementation is fixed;
\item There can be maximal one superimposition and implementation part.
\end{itemize}

%\faq{
%\para{FAQ}
%\para{Hints}
%}
\comments{\para{Different Usages of a Concern}
The \Compose* concern has a multi-role usage due to the way it is built up.
The different combinations of concern elements and their explanations are shown in \autoref{tab::arm::con:concernusage}~\cite{bergmans:aosdbook05}.
The usages in this table are all the possible usages of concerns. 

\begin{table*}[tpb]
\begin{center}
	\centering
	% table : FM - SI - Impl -- Ex
		\begin{tabular}{I c | c | c I l I} \hline
		Filter& Super- & Implemen- &  \\
		Module(s)& imposition & tation & Explanation\\ \hline
		No  & No	& Yes	& CF conventional class\\ \hline
	  Yes & only to self & Yes & CF conventional CF class\\ \hline
	  Yes & Yes & Yes	& Crosscutting concern with implementation\\ \hline
	  Yes & Yes & No	& ``Pure'' crosscutting concern, no implementation\\ \hline
	  Yes & No  & No 	& CF abstract advice without crosscutting definition\\ \hline
	  No  & Yes	& No 	& Superimposition only (of reused filter specs.)\\ \hline
	  No  & Yes	& Yes & CF class or aspect with only reused filter specs.\\ \hline
	 	\end{tabular}
	\caption{Different concern usages}
	\label{tab::arm::con:concernusage}
\end{center}
\end{table*}

\para{Concern Parameters}
There has been the idea to use global variables in a concern. 
These variables were declared in a concern with the concern parameters block. 
\autoref{lst::ARM:concern:example1} shows an example usage of concern parameters.
However it became clear that it is not possible to assign these parameters anywhere in the application.
With the introduction of superimposition it is no longer useful to instantiated concerns manually with arguments.
Therefore the concern parameters are not implemented.

\begin{lstlisting}[caption={JoinPointInfoConcern from the JoinPointInformation example },label=lst::ARM:concern:example1,style=listing,language=ComposeStar,float=[tpb]]
concern JoinPointInfoConcern
    (ext_foo : JPInfo.Foo ; ext_bar : JPInfo.Bar) in JPInfo{

  filtermodule JPInfoModule{
    internals
      in_foo : JPInfo.Foo;
      in_bar : JPInfo.Bar;
      jpie   : JPInfo.JoinPointInfoExtractor;
    conditions
      cond : jpie.doCondition();
    inputfilters
      meta : Meta = { cond => [*.*] jpie.extract }
  }

  superimposition{
    selectors
      selectees = { C | isClassWithName(C, 'JPInfo.JoinPointInfoConcern') };
    filtermodules
      selectees <- JPInfoModule;
  }

  implementation by JPInfo.JoinPointInfoConcern;
}
\end{lstlisting}
}
\dotnetcomment{\para{Concern Parameters}
\begin{lstlisting}[caption={JoinPointInfoConcern from the JoinPointInformation example },label=lst::ARM:concern:example1,style=listing,language=ComposeStar,float=[tpb]]
concern JoinPointInfoConcern
    (ext_foo : JPInfo.Foo ; ext_bar : JPInfo.Bar) in JPInfo{

  filtermodule JPInfoModule{
    internals
      in_foo : JPInfo.Foo;
      in_bar : JPInfo.Bar;
      jpie   : JPInfo.JoinPointInfoExtractor;
    conditions
      cond : jpie.doCondition();
    inputfilters
      meta : Meta = { cond => [*.*] jpie.extract }
  }

  superimposition{
    selectors
      selectees = { C | isClassWithName(C, 'JPInfo.JoinPointInfoConcern') };
    filtermodules
      selectees <- JPInfoModule;
  }

  implementation by JPInfo.JoinPointInfoConcern;
}
\end{lstlisting}
As mentioned in the Commentary of the concerns the concern parameters are removed from the \Compose* syntax.
This change only exist in the design and in \Compose*[DOTNet] they still exist in the syntax. It is however not clear
whether they work or not. In \autoref{lst::ARM:concern:example1} the concern is taken from the JoinPointInformation example,
which can be found in the \Compose* example directory. In this concern specification the concern parameters are declared, but never used in a filter module.}
%\javacomment{}
%\ccomment{}
%\pending{}
\furtherreading{
In the comments section of this chapter \cite{bergmans:aosdbook05} is already mentioned, this is a good source for the theory behind concerns.
}