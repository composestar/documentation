\para{Concern Parameters}
\begin{lstlisting}[caption={JoinPointInfoConcern from the JoinPointInformation example },label=lst::ARM:concern:example1,style=listing,language=ComposeStar,float=[tpb]]
concern JoinPointInfoConcern
    (ext_foo : JPInfo.Foo ; ext_bar : JPInfo.Bar) in JPInfo{

  filtermodule JPInfoModule{
    internals
      in_foo : JPInfo.Foo;
      in_bar : JPInfo.Bar;
      jpie   : JPInfo.JoinPointInfoExtractor;
    conditions
      cond : jpie.doCondition();
    inputfilters
      meta : Meta = { cond => [*.*] jpie.extract }
  }

  superimposition{
    selectors
      selectees = { C | isClassWithName(C, 'JPInfo.JoinPointInfoConcern') };
    filtermodules
      selectees <- JPInfoModule;
  }

  implementation by JPInfo.JoinPointInfoConcern;
}
\end{lstlisting}
As mentioned in the Commentary of the concerns the concern parameters are removed from the \Compose* syntax.
This change only exist in the design and in \Compose*[DOTNet] they still exist in the syntax. It is however not clear
whether they work or not. In \autoref{lst::ARM:concern:example1} the concern is taken from the JoinPointInformation example,
which can be found in the \Compose* example directory. In this concern specification the concern parameters are declared, but never used in a filter module.