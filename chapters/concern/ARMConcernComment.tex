\para{Different Usages of a Concern}
The \Compose* concern has a multi-role usage due to the way it is built up.
The different combinations of concern elements and their explanations are shown in \autoref{tab::arm::con:concernusage}~\cite{bergmans:aosdbook05}.
The usages in this table are all the possible usages of concerns. 

\begin{table*}[tpb]
\begin{center}
	\centering
	% table : FM - SI - Impl -- Ex
		\begin{tabular}{I c | c | c I l I} \hline
		Filter& Super- & Implemen- &  \\
		Module(s)& imposition & tation & Explanation\\ \hline
		No  & No	& Yes	& CF conventional class\\ \hline
	  Yes & only to self & Yes & CF conventional CF class\\ \hline
	  Yes & Yes & Yes	& Crosscutting concern with implementation\\ \hline
	  Yes & Yes & No	& ``Pure'' crosscutting concern, no implementation\\ \hline
	  Yes & No  & No 	& CF abstract advice without crosscutting definition\\ \hline
	  No  & Yes	& No 	& Superimposition only (of reused filter specs.)\\ \hline
	  No  & Yes	& Yes & CF class or aspect with only reused filter specs.\\ \hline
	 	\end{tabular}
	\caption{Different concern usages}
	\label{tab::arm::con:concernusage}
\end{center}
\end{table*}

\para{Concern Parameters}
There has been the idea to use global variables in a concern. 
These variables were declared in a concern with the concern parameters block. 
\autoref{lst::ARM:concern:example1} shows an example usage of concern parameters.
However it became clear that it is not possible to assign these parameters anywhere in the application.
With the introduction of superimposition it is no longer useful to instantiated concerns manually with arguments.
Therefore the concern parameters are not implemented.

\begin{lstlisting}[caption={JoinPointInfoConcern from the JoinPointInformation example },label=lst::ARM:concern:example1,style=listing,language=ComposeStar,float=[tpb]]
concern JoinPointInfoConcern
    (ext_foo : JPInfo.Foo ; ext_bar : JPInfo.Bar) in JPInfo{

  filtermodule JPInfoModule{
    internals
      in_foo : JPInfo.Foo;
      in_bar : JPInfo.Bar;
      jpie   : JPInfo.JoinPointInfoExtractor;
    conditions
      cond : jpie.doCondition();
    inputfilters
      meta : Meta = { cond => [*.*] jpie.extract }
  }

  superimposition{
    selectors
      selectees = { C | isClassWithName(C, 'JPInfo.JoinPointInfoConcern') };
    filtermodules
      selectees <- JPInfoModule;
  }

  implementation by JPInfo.JoinPointInfoConcern;
}
\end{lstlisting}
