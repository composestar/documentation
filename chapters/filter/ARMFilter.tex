% ARM entry -> Filter
\chapter{Filters} \label{chapter:filter}
Filters are the main part of a filter module, a filter module can have input filters and output filters.
Both the input filters and output filters have the same syntax and therefore we handle them both in this chapter.

A filter has five parts: the filter identifier, the filter type, the condition part, the matching part, and the substitution part. These parts are shown below:

\begin{center}
$\overbrace{stalker\_filter}^{identifier}:\overbrace{Dispatch}^{filter~type}~=~\{\overbrace{!pacmanIsEvil}^{condition~part}
=>\overbrace{[*.getNextMove]}^{matching~part}~\overbrace{stalk\_strategy.getNextMove}^{substitution~part}~\}$
\end{center}

Except for the filter identifier they all are separately described in the reference manual. In this chapter we look to how these parts work together and how sequential filters behave.

\section*{Syntax} %\label{insert_label}
\begin{lstlisting}[caption = {Filter syntax}, label = lst::ARM:fil:syntax,
style = listing, language = ebnf, float = tpb]
FilterModule ::= `filtermodule' FilterModuleName [FilterModuleParameters] `{'
                  [Internals] [Externals] [Conditions]
                  [InputFilters] [OutputFilters] `}'
Inputfilters ::= `inputfilters' FilterSet
Outputfilters ::= `outputfilters' FilterSet

FilterSet ::= Filter (FilterOperator Filter)*
FilterOperator ::= `;'
Filter ::= FilterName `:' FilterType [`(' ArgumentList ')'] `=' 
           `{' FilterElements `}'
FilterElements ::= FilterElement (ElementCompositionOperator FilterElement)*
ElementCompositionOperator ::= ','

FilterElement ::= [ORExpression ConditionOperator] MessagePattern
ORExpression ::= ANDExpression [`|' ANDExpression]
ANDExpression ::= NOTExpression [`&' NOTExpression]
NOTExpression ::= [!] (ConditionLiteral | `('ORExpression`)')
ConditionLiteral ::= ConditionName | `True'| `False'
ConditionOperator ::= `=>'| `~>'
MessagePattern ::= Matching [SubstitutionPart]
                   | MatchPattern
                   | `{' Matching (`,' Matching)* `}' [SubstitutionPart] 
Matching ::= SignatureMatching | NameMatching
SignatureMatching ::= `<' MatchPattern `>'
NameMatching ::= `[' MatchPattern `]' | Quote MatchPattern Quote
SubstitutionPart ::= [Target `.'] Selector
MatchPattern ::= [Target `.'] Selector
\end{lstlisting}

The syntax in \autoref{lst::ARM:fil:syntax} covers the full syntax of the filter, so it also covers the condition, matching, and substitution part.
The filter type contains the type of the filter, this can be a predefined one (currently Dispatch, Meta, Send, or Error) or a custom type.
A filter element
contains an optional condition part and
one of more message patterns. The conditional part contains a logical statement, possibly containing logical operators in combination with conditions and boolean values. The message pattern contains a name or signature matching and an optional substitution part. The filters are separated with a filter operator. Currently the only operator is the ``;''.

\section*{Semantics}
As mentioned earlier, a filter can consist of five parts. The filter identifier can be used to get the corresponding filter,
like other filter module elements. The filter element is the combination of condition, matching, and substitution part.
How the filter element behaves, depends on the filter type. If the condition and matching part matches then the
substitution part gets executed.

Filters are separated by a ``;'', this operator means that if the filter does not match, the next filter in the filter module will be evaluated. After the last superimposed filter module a \emph{default dispatch filter} to the \emph{inner object} is created. This means that the last filter that is declared in a filter module is always followed by another filter, be it from the next filter module or it is the default dispatch filter. To keep the user from placing another filter operator between these couplings, it is not possible to use the ``;'' after the last filter in a filter module.%b$

Input filters reason about messages going to the object and output filters on the messages that are sent from an object. The return call of a method is not considered a message.

\section*{Examples}
\begin{lstlisting}[caption = {Dynamic strategy filter module in Pacman}, label=lst::arm::fil:example1,style = listing, language = ComposeStar,float=[tpb]]
filtermodule dynamicstrategy{
  internals
    stalk_strategy : pacman.Strategies.StalkerStrategy;
    flee_strategy : pacman.Strategies.FleeStrategy;
  conditions
    pacmanIsEvil : pacman.Pacman.isEvil();
  inputfilters
    stalker_filter : Dispatch = {!pacmanIsEvil => 
      [*.getNextMove] stalk_strategy.getNextMove };
    flee_filter : Dispatch = {[*.getNextMove]flee_strategy.getNextMove }
}
\end{lstlisting}

When we look at the dynamic strategy concern of Pacman, \autoref{lst::arm::fil:example1}, we have two filters: stalker filter and flee filter. Both filters are of the filter type dispatch, this
means that if the condition part and matching part matches, the target and selector gets substituted with the values in the substitution part and the message is then dispatched to the target of the changed message.

The combination of the two filters means that the stalker filter is evaluated first and if it does not match then the next filter is being evaluated, this is the flee filter in this situation.
The matching is based on the condition part, in the example this is the condition whether Pacman is evil,
and the matching part, which filters on messages with ``getNextMove'' as selector value.
When a filter matches, the target and selector of the message gets changed with the values of the substitution part. What happens with the message then depends on the filter type.
To specify how sequential filters behave there is an filter operator placed between the filters. Currently only the ``;'' is used and it means ``if not then''.  It is not possible to place an operator after the last filter in a filter set.

\begin{lstlisting}[caption = {Custom logging filter module}, label=lst::arm::fil:example2,style = listing, language = ComposeStar,float=[tpb]]
filtermodule logger(??inputMethods, ??outputMethods){
  externals
    logger : Logger = Logger.instance();
  conditions
    isLoggingOn : logger.isOn();
  inputfilters
    inlog : Meta = {isLoggingOn => [*.??inputMethods] logger.log}
  outputfilters
   outlog : Meta = {isLoggingOn => [*.??outputMethods] logger.log}
}
\end{lstlisting}

An example on how you can place input and output filters in one filter module is demonstrated in \autoref{lst::arm::fil:example2}. In the example the working of both filters is the same, if the condition
matches and the selector is in the set of the given methods, then the message getss wrapped up and is send to the external object and to be more precise to the method \lstinline|log|.

\section*{Legality Rules}
\begin{itemize}[noitemsep]
\item The identifier of a filter must be unique for the filter module where it is declared in, so it is not possible to have the same name for an input and output filter;
\item Filters are separated by a ``;'', which is called the composition operator;
\item The last filter is not followed by a ``;''.
\end{itemize}

\faq{
\para{FAQ}
\question{Why is it not possible to place a ``;'' after the last filter?}
\answer{The ``;'' between filters is often mistaken for a terminator symbol, but it actually says something
about the sequential ordering of filters. The last filter in a filter module is always followed by the filter of the next filter module or the default dispatch filter. It would be incorrect to let the user have a say about that coupling and therefore it is not possible to place a ``;'' after the last filter.}

\para{Hints}
To keep a message from being evaluated by the default dispatch filter it is possible to place a last filter that catches all
messages. This could be an error or a dispatch filter.
}
\comments{
\para{Filter Identifier}
The filter has a unique identifier which gives the opportunity to get a filter from its filter module by
using a filter module element reference. Currently this construct is not used in \Compose* and it comes from the original idea to reuse filter specifications.
So at the moment the identifier is redundant, but we keep it in the syntax for usability and user-friendliness. Because talking about the fleeStrategy filter is easier
than talking about the first filter of filter module dynamic strategy.

\para{Filter Modules with both Input and Output Filters}
As mentioned earlier it is possible to have input and output filters in one filter module. However, the option of using both input and output filters in a filter module is barely used.
One of the reasons for this is that for most usages of the filter module you only need one of the two filter sets. Some of the used constructs for filters only need an input filter, like for instance the dynamic strategy filter module in Pacman (\autoref{lst::arm::fil:example1}) or a generic inheritance filter module (\autoref{lst::ARM:fmp:example2}).

\begin{figure}[tpb]
	\centering
	\includegraphics[style=thirdheight]{ARM-filter-constraints}
	\caption{Two filter modules with input and output filters}
	\label{fig::arm::fil:comment1}
\end{figure}

A limitation for having both input and output filters in one filter module is how the constraints are defined. For instance, if we take two filter modules logging and inheritance and they both have input and output filters and we want to log before the input filters of inheritance and we want to log after the output filters of inheritance, this is demonstrated on the left side of \autoref{fig::arm::fil:comment1}. However in reality
you end up with the situation on the right, and the only way to get it like the left image is when you break up one of the two filter modules.
To avoid situations like this, it is advisable to only combine input and output filters if it is really necessary, for instance if they share an internal.

\para{Filters as State Machines}
\begin{lstlisting}[caption = {Idea for a filter state machine}, label=lst::arm::fil:comment1,style = listing, language = ComposeStar,float=[tpb]]
inputfilters
  first : Dispatch = {[*.log]} REPEAT
  second : Dispatch = {[*.print]}
}
\end{lstlisting}
\begin{lstlisting}[caption = {State machine with filter operators}, label=lst::arm::fil:comment2,style = listing, language = ComposeStar,float=[tpb]]
internals
  state = State;
conditions
  ifRaisedState : state.ifRaisedState();
inputfilters
  first : Before = {[*.log] *.raiseSate};
  second : Dispatch = {ifRaisedState => [*.print]}
}
\end{lstlisting}
An idea for the usage of the filter operators is to use them for building a state machine. An example is given in
\autoref{lst::arm::fil:comment1}, in this example we want to let the first filter evaluate and when it matches it moves to the second one until that one matches.
Whereas this looks like a nice idea, but there are drawbacks. First you get
problems with the
syntax and semantics, because you must be able to scope your REPEAT statement and you also need to define
what the commando's mean. Second, designing such filter sets is hard and it is much better to write
such a machine with an internal that keeps the state in combination with conditions that handle the state, like \autoref{lst::arm::fil:comment2}.
The filter of the message for the state change can be done with a meta filter or even the prepend and append filter~\cite{Minnen2006}. Writing such construction is easier then using state machine operators.

\para{Filter Operators and Different Filter Combinations}
The filter operator is placed between filters. At the moment only the ``if not then'' operator is used which is in the ``;'' token in the syntax. There are two points to comment on the filter
operator. The first question is why the usage of ``;''? And the second question is can the filter operator mean anything at all?

To start with the second question, as mentioned above with designing the state machine it hard to define new filter operators because you probably end up adding parenthesis and other syntax constructs. Therefore there is no final answer on this question yet, there are some ideas on more filter operators around. But these ideas have problems on the what the semantics are of these operators.
And to answer the first question, it is indeed a bit strange to use the semicolon as an operator. In most programming languages it marks the end of a statement. We do not allow
a semicolon after the last filter declaration and this often regarded as annoying by the users. We could add the ``+'' sign as a stand in for ``;'' to make things less confusing. We have not done it yet, due to the fact that we cannot answer the second question yet and it would
be strange to start adding operators while we have not answered some of the problems with combining filters.

}
%\dotnetcomment{}
%\javacomment{}
%\ccomment{}
%\pending{}
%\furtherreading{}