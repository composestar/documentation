% empty template for ARM entry. By Dirk Doornenbal. V1.2 18 april 2006
% Internals entry V1.0 ?? april 2006
\nomenclature{LHS}{Left Hand Side}%
\nomenclature{RHS}{Right Hand Side}%
\nomenclature{EBNF}{Extended Backus-Naur Form}%
\nomenclature{OO}{Object Oriented}%
\chapter{Internals}%\label{insert_label}
Internals are the object instances that are instantiated for each instance of a filter module. 
Because of this, internals can be used in situations where each instance of a filter module must have its own instance of an object, for example to hold a state or for inheritance by delegation.

\section{Syntax}
The internal declaration has two parts, which are separated with a colon. 
The left hand side contains the identifiers of the internals, and the right hand side the type of those identifiers. 
The formal syntax is defined in \autoref{lst::ARM:int:syntax}.

The types are defined by fully qualified names.
You can refer to an internal in the conditions field, the filter parameters, and the Target.
Internals can only be objects, primitive types are not allowed.
\begin{lstlisting}[caption = {Internals syntax}, label = lst::ARM:int:syntax,
style = listing, language = ebnf, float = tpb]
FilterModule = 'filtermodule' Identifier [FilterModuleParameters] '{' [Internals] [Externals] [Conditions] [InputFilters] [OutputFilters] '}';
Internals = 'internals' {Internal ';'};
Internal = Identifier {',' Identifier} ':' Type;
\end{lstlisting}

\section{Semantics}
When the filter module gets instantiated, all its internals get instantiated using the default constructor.
\autoref{lst::ARM:int:example2} shows the binding of a filter module \emph{isActive} to classes A and B. 
The result of this is shown in \autoref{fig::arm::int:initialisation}.
Every instance of A and B gets its own instance of the filter module \emph{isActive} and each instance of the filter module gets its own instance of the internal \lstinline!state!.

The effect is that every instance of an object gets its own instance of the filter module and thus indirect the internal, making it suitable for storing states independently and for inheritance by delegation. 
With the proper filter construction, like \lstinline!d : Dispatch = { <internal.*> internal.*}!, it is possible to extend the signature of a superimposed object with the signature of the internal.
This means that you do not need to access the internal from outside the filter module, because you can directly
use the object on which is superimposed.

\begin{lstlisting}[caption = {Example 2, a filter module to hold a state}, label = lst::ARM:int:example2,
style=listing, language=ComposeStar, float = tpb]
filtermodule isActive{
  internals
    state : State;
  conditions
    notActive : state.notActive();
  inputfilters
    d : Dispatch ={<state.*> state.*};
    e : Error (state) = {notActive => [*.*]}
}
superimposition{
  selectors
    sel = {C | isClassWithNameInList(C,['A', 'B'])};
  filtermodules
    sel <- isActive;
}
\end{lstlisting}

\begin{figure}[tpb]
	\centering
	\includegraphics[style=thirdheight]{images/ARM-internalInitialization}
	\caption{Instances of \autoref{lst::ARM:int:example2}}
	\label{fig::arm::int:initialisation}
\end{figure}

\section{Examples}
In the Pacman example (\autoref{lst::ARM:int:example1}), the strategies are internals; this gives that they are instantiated for each filter module. 
An internal is declared with a fully qualified name of a class.

\begin{lstlisting}[caption = {Example 1, a piece of Pacman}, label = lst::ARM:int:example1,
style=listing, language =ComposeStar, float = tpb]
filtermodule dynamicstrategy{
  internals
    stalk_strategy : pacman.Strategies.StalkerStrategy;
    flee_strategy : pacman.Strategies.FleeStrategy;   
  conditions
    ...
\end{lstlisting}

\section{Legality Rules}
\begin{itemize}[noitemsep]
\item The identifier of an internal must be unique for the filter module where it is declared in;
\item The type must be a fully qualified name of a class and this class must exist;
\item In order to be instantiated, the class for the internal must have a public constructor without parameters;
\item When an internal is declared but not used you get a warning from the compiler.
\end{itemize}

\faq{
\para{FAQ}
\question {What is the use of declaring multiple instances of one type? Is it not possible to rewrite the class of the type, so that you only need to declare one internal?}
\answer {This depends on the situation, whereas for holding a state you can also write another class that takes care of the complete state, however for constructions where you delegate to a tree-like structure, like \lstinline!left, right : Node;!, you do not want to combine the internals into one class.}

\question {Is it possible to access an internal of another filter module?}
\answer {It is not possible, on conceptual and implementation level, to access an internal with a filter module element reference.}

\question {I want to declare an internal with a type that is in a library, but the class does not have a default constructor. 
How can I solve this?}
\answer{
The best way to solve this is by extending the class you want to use with a default constructor that sets the needed default values. 
If that is not possible because the class has been made final, then you can write a static method which calls a constructor with arguments and returns the result, this static method can be used in the declaration of an external\footnote{\nb: Only use this construction if you are stuck in such particular situation, see also the commentary.}.
}
 
\para{Hints}
The instantiation of an internal happens with a constructor with no parameters. 
In most object-oriented languages this is the default constructor, which is used when the user does not create a constructor. 
Sometimes people forget to create a constructor without parameters, when there is already a constructor with parameters, which results in an error.
}

\comments{\para{The use of Arguments in Constructors Calls}
The internal initialization is done with a constructor without arguments;
this is a result of the requirement to be language independent. The internal declaration 
\lstinline!internal: Person = Person(``Albert'', 24, true);! is considered as language dependent and therefore
only constructors without arguments are used. We will not re-discuss the point of allowing primitive values
in the internal declaration, because is has been discussed in \autoref{chapter:filtermodule}.
However, with the introduction of the filter module parameters it is possible to
use objects, which are passed through the filter module parameters, in the declaration of the internals.
This feature is not added to the syntax yet, because it relies on the full implementation of the filter
module parameters. When this is realized then the syntax can change to the one given in \autoref{lst::ARM:int:syntax2}.
\\\\
\begin{lstlisting}[caption = {Proposed internal syntax}, label = lst::ARM:int:syntax2,
style = listing, language = ebnf]
Internals ::= `internals' (Identifier-LIST `:' Type 
              [`=' InitialisationExpression `(' [ Argument-LIST ] `)']`;')*
\end{lstlisting}
Because we already use default constructors, we already have the mapping from the language independent declaration to language specific constructor calls, thus we do not have to work them out for adding arguments in a constructor.}

\dotnetcomment{If you really want to access an internal then there are two options, first you can access the filter information from the repository, which has at runtime the reference to the internal you want. 
%\footnote{As you might guessed this is actually a hack.}.
Second, it is possible to write a filter that returns the internal and with the extension of the signature of the superimposed object you can get a \emph{get-method} for the internal, this construction is demonstrated in \autoref{lst::ARM:fmp:commentdotnet1}. 
The second option is preferable above the first one, because it is more clear of what you do, only in certain cases where you cannot extend the signature you are stuck with the first option.

\begin{lstlisting}[caption = {How to get hold of an internal}, label = lst::ARM:fmp:commentdotnet1,
style = listing, language=ComposeStar, float = tpb]
filtermodule returnInternal{
  internals
    state : State;
  inputfilters
    d : Dispatch = {[*.getState] s.getSelf}
}
\end{lstlisting}}

%\ccomment{To be specified\\\\\\\\\\}

%\pending{}

%\furtherreading{}

