% Condition part ARM entry. By Dirk Doornenbal. V1.5 2 mei 2006
\chapter{Condition Expression} \label{chapter:conditionpart}
The condition part of a filter contains the conditions and operators which say something of the matching part, to reason whether a filter is accepted or not. 
The used methods must be first declared in the conditions block of a filter module. 
The logical operators are the basic set of \emph{and}, \emph{or}, and \emph{not} added with two operators to denote.

\section{Syntax}
The syntax of the condition part, as demonstrated in \autoref{lst::ARM:cp:syntax}, follows the standard way to
program a set of AND (\lstinline|&|), OR (\lstinline$|$), and NOT(\lstinline|!|) operators. 
The precedence is that the NOT is executed first, then the AND and the OR.
This means that \emph{A AND B OR C} is actually \emph{(A AND B) OR C}. 
The chosen precedence is the commonly used one.
Parenthesis can be used to group subexpressions.

\begin{lstlisting}[caption={Filter condition part syntax},label=lst::ARM:cp:syntax,style=listing,language=ebnf,float=tpb]
ConditionExpression = AndExpression ['|' ConditionExpression];
AndExpression = UnaryExpression ['&' AndExpression];
UnaryExpression = ['!'] ( Operand | '(' ConditionExpression ')' );
Operand = Identifier | 'True' | 'False';
\end{lstlisting}

The \emph{Operand} can be either a condition name, which must be declared in the condition block of a filter module, or one of the constants ``True'' or ``False''. 
The constants are case insensitive.

\section{Semantics}
In the condition part it is only possible to use the boolean values \emph{true} and \emph{false} or conditions that are declared in the conditions block of a filter module.
This means that the number of possible syntax constructs is small.
The meaning of the logical operators is the same as the usual meaning in most programming languages.
The condition expression is evaluated before the message is matched to the matching part.
This means that the message will not be matched when the result of the expression is false. 

The condition part is optional, when it is omitted the constant expression \lstinline|True| is used.

\section*{Examples}
In the Pacman example of \cite{Composestar} we have already seen the filter demonstrated in \autoref{lst::armcp:example1}. 
In that example the method \emph{pacmanIsEvil} is checked in order to determine how the ghosts must behave. 
So when Pacman is not evil it only depends on the matching part whether the filter matches.

\begin{lstlisting}[caption={The dymanic strategy filtermodule}, label=lst::armcp:example1,style=listing,language=Composestar,float=[tpb]]
conditions
  pacmanIsEvil : pacman.Pacman.isEvil();
inputfilters
  stalker_filter : Dispatch = {!pacmanIsEvil => 
                              [*.getNextMove] stalk_strategy.getNextMove};
  flee_filter : Dispatch = {[*.getNextMove] flee_strategy.getNextMove}
\end{lstlisting}

Some examples of condition parts are given in \autoref{lst::armcp:exampl2}. 
We see that d1 (on line five) depends on the condition isBlue whether the message will be accepted.
Filter d2 (on line six) shows a combination of operators and values.

\begin{lstlisting}[caption={Possible condition parts}, label=lst::armcp:exampl2,style=listing,language=Composestar,float=[tpb]]
conditions
  isBlue : inner,isBlue();
  isMorning : inner.isMorning();
inputfilters
  d1 : Dispatch = {!isBlue => *.*};
  d2 : Dispatch = {(isBlue | isMorning) & True => *.*}
\end{lstlisting}

\section{Legality rules}
\begin{itemize}[noitemsep]
\item A condition name must be declared in the condition block of the same filter module;
\item The keywords true and false are case insensitive.
\end{itemize}

\faq{
\para{Hints}
Whereas the condition part is optional, the matching part is not. 
It is however easy to let only a condition part determine the outcome of a filter, with the matching \lstinline|[*.*]| every message matches and it only depends on the conditions whether a message is accepted.
}

\comments{\para{Basic Set of Logical Operators}
The supported set operators are the \emph{and}, \emph{or}, and \emph{not}. These three are enough to create any other possible logic operator because they form a \emph{functional complete} set~\cite{vanBenthem91}.
Because we want to reason about the filters, we keep the syntax as simple as possible on this point. It is easier to write an algorithm that only has to take care of three
operators instead of a whole set.

We could go further by removing all the operators and just write one method that calls all the separate
conditions, however this would also mean that this method should also know the references to the internals, externals,
and inner object, and even more important this method would be quite complex and not be reusable at all. Therefore we adopt a limited set
of logical operators.

%\dotnetcomment{
%As of 0.6 this is properly implemented
%\para{Current Implementation}
%Currently the condition part is not programmed as described in this chapter; the nesting is not in the logical operators yet.
%Thus \lstinline|(True & False) & True| will give an error. Because the \Compose* grammar is the same for all ports it therefore
%affects every port.
%}
}
%\dotnetcomment{}
%\javacomment{}
%\ccomment{}
%\pending{}
%\furtherreading{}
