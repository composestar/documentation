% ARM entry for the conditions.
\chapter{Conditions} \label{ch::arm:con}
The conditions are used to introduce (boolean) methods  in a filter module so that these methods can be used in a filter specification
to influence the behavior of a filter at runtime. To keep the filter specification simple and to reuse declarations, all the conditions of a filter module are declared in one place: the conditions declaration block.
The methods for implementing the conditions can be defined in the inner object (the class where the filter module is superimposed to), internals, externals, or are static methods.

\section*{Syntax}
\begin{lstlisting}[caption={Conditions syntax},label=lst::ARM:con:syntax,style=listing,language=ebnf,float=[tpb]]
FilterModule ::= `filtermodule' FilterModuleName [FilterModuleParameters] `{'
                  [Internals] [Externals] [Conditions]
                  [InputFilters] [OutputFilters] `}'
Conditions ::= `conditions' (Identifier `:' MethodReference `;')*
MethodReference ::= (FilterModuleElement``.''MethodName | FullyQualifiedName)
                    `(' `)'
\end{lstlisting}
From the syntax, \autoref{lst::ARM:con:syntax}, we see that the condition name and declaration are a one-to-one relation. Each condition name must be unique for each filter module and the name may also not be used for an internal or external. There are two forms of declaration: a condition can be declared from
an internal, external, or the inner object and the other option is to use a fully qualified name of a static method.

\section*{Semantics}
A condition declaration labels a method reference with a identifier so that you can use this identifier in the filter specification, instead of a method reference. The method reference of the condition declaration needs to point to a method that returns a boolean.

\section*{Examples}
\begin{lstlisting}[caption={Some condition declarations combined},label=lst::ARM:con:example1,style=listing,language=ComposeStar,float=[tpb]]
filtermodule someConditions{
  internals
    a : VenusFlyTrapExample.Animal;
  externals
    jbframe : JukeboxFrame.JBFrame = JukeboxFrame.JBFrame.instance();
  conditions
    isfly : a.hasPrey(); // Venusfly trap
    isStateChanged : jbframe.isStateChanged(); //Jukebox
    pacmanIsEvil : pacman.Pacman.isEvil(); //Pacman
    isFrame : Composestar.Patterns.ChainOfResponsibility.Click.isFrame();
    // Command of responsibility pattern
    isBlue : inner.isBlue();
\end{lstlisting}
In \autoref{lst::ARM:con:example1} five examples of condition declarations are given. Four of these declarations come from the \Compose* examples~\cite{Composestar}, for each of them is noted in the listing from which example they come. 
The first one shows the usage of an internal, the second one shows the usage of an external.
The third and fourth examples show the use of static methods.
The last example does not come from the example directory, there is a design technical issue why this option is not much used.
There is no example that uses all the different condition declarations in one condition declaration block.

\section*{Legality rules}
\begin{itemize}[noitemsep]
\item The identifier of a condition must be unique for the filter module it is declared in;
\item The condition implementation method that you reference to must exist and it must return a boolean value;
\item A condition implementation method must not have any side effects in the application.
\end{itemize}

%\faq{
%\para{FAQ}
%\question{}
%\answer{}
%\para{Hints}
%} clear chapter no faq according to me...

\comments{\para{Parenthesis and Arguments}
The current implementation does not allow the usage of arguments in the condition declaration.
As mentioned in~\cite{Doornenbal2006}
it is a possible addition to the language to add arguments in the condition declaration, because of the introduction of the filter module parameters.

\para{The Use of Conditions from the Inner Object}
One of the possibilities to declare a condition is to use a boolean method of the inner object. 
As mentioned earlier, this option is not used in the current set of examples. 
We can say that it is not widely used due to two limitations:
the methods must exist and you should know this when you write the selector. 
Filter modules are imposed on multiple classes and when a method of the inner class is used, then all the multiple classes must contain this method.

There are three ways to use a condition from an inner object without getting any problems with non-existing methods. 
The first is to write filter modules for just one class or its child classes. 
The second use for conditions from the inner object is if you select all the classes which implement a certain interface. 
In that scenario you know whether the boolean method is available if it is in the signature of the interface.
The third one is code conventions, if you use a code convention that classes with a certain annotation all have a method called \emph{foo}, then it is also possible to use the conditions from the inner object safely.

\para{Conditions bound to Static Methods}
It is currently not possible to bind conditions to static methods, this functionality has not be implemented.

\para{Language Independent Conditions}
The selector in the superimposition uses predicate queries to select certain attributes of an application. 
It uses a language independent model of language constructs and each language gets translated to this independent model~\cite{Havinga2005}. 
For instance it uses the term namespace, which gets translated to package for the Java language. 
In theory it is possible to write such language independent model on objects and its attributes, so that we can write language independent conditions with predicate queries. 
This would make it possible to write conditions that are reusable even for another platform.}
%\dotnetcomment{}
%\javacomment{}
%\ccomment{}
%\pending{}
%\furtherreading{}