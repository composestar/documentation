% empty template for ARM entry. By Dirk Doornenbal. V1.5 2 mei 2006
\chapter{Implementation Part} \label{chapter:implementation}
The implementation part is were the base language dependent code of a concern is specified. The code
is copied to a format known to the base language and is compiled by the compiler of the base language. This
is done to save the effort of writing our own compiler.

\section*{Syntax}
\begin{lstlisting}[caption = {Implementation syntax}, label = lst::ARM:impl:syntax,
style = listing, language = ebnf, float = tpb]
Implementation ::= `implementation'(`by' ClassName `;'
                   | `in' SourceLanguage `by' ClassName `as' FileName `{' Source `}'
Source ::= (?)*
\end{lstlisting}
The heading of the implementation contains the name of the source language, the class name and the file name.
The file name needs to be marked with quotes.
The source code must be written according the specification of the base language. The \Compose* syntax for
the implementation part is shown in \autoref{lst::ARM:impl:syntax}.

\section*{Semantics}
The implementation part is the base language dependent part of a concern. As we have already seen in \autoref{ch::arm:con}
is a concern with only an implementation part just a class.

\section*{Examples}
\autoref{lst::ARM:ec:example1} shows a C\# class that is fully placed in a concern.
Besides adding a complete class by stating it in the \emph{cps} file it also possible to point to a dll file.
Every concern can only have one implementation part.

\begin{lstlisting}[caption={The class Bar as concern},label=lst::ARM:ec:example1, style=listing,language=ComposeStar,float = tpb]
concern Bar in GrammarTest{
  implementation in Csharp by Bar as "Bar.cs"{
    using System;
    
    namespace GrammarTest{
      public class Bar{
        public void write(){
          System.Console.WriteLine("Bar");
        }

      public bool isBar(){
        return true;
      }
    }
  }
}
\end{lstlisting}

\section*{Legality rules}
\begin{itemize}[noitemsep]
\item When the code is added in a concern, you need the full heading;
\item The extension of the file name must be appropriate for the base language.
\end{itemize}

%\faq{
%\para{Hints}
%}
\comments{
There have been several ideas on the usage of the implementation part. In \cite{salinas:ms01} the proposal is made to
add implementations for each base language. So for the source for a logger you write for instance a Java and a C$\sharp$
source.
}
\dotnetcomment{
The code that is placed in this block, is copied
to the directory embedded, in the file with the name you have specified.

The syntax cannot handle ``$\sharp$'' in the source language, therefore you need to write sharp full out.
}
%\javacomment{}
%\ccomment{}
%\pending{}
%\furtherreading{}