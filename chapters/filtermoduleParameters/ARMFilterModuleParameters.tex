% empty template for ARM entry. By Dirk Doornenbal. V1.1 17 april 2006
% Filtermodule parameters entry V1.0 22 april 2006
\chapter{Filter Module Parameters}
Filter module parameters can be used to bind variables to a filter module when the filter module is bound to an object.
It is possible to use a single parameter value or a list of parameter values; these will be referred to as \emph{single parameter} and \emph{list of parameters}.
The scope of the filter module parameters is the filter module where they are declared.

\section{Syntax}
Parameters declarations are placed right after the name of the filter module. 
A parameter must have a ``?'' or a ``??'' prefix, followed by an identifier.
The single question mark is for a single value and the double question mark is to mark a parameter list.
After you have declared the parameters you can use them in the filter module. The formal syntax is stated in \autoref{lst::ARM:fmp:syntax}.

\begin{lstlisting}[caption = {Parameter declaration syntax}, label = lst::ARM:fmp:syntax,
style = listing, language = ebnf, float = tpb]
FilterModule = 'filtermodule' Identifier [FilterModuleParameters] '{' [Internals] [Externals] [Conditions] [InputFilters] [OutputFilters] '}';                                    
FilterModuleParameters = '(' ParameterDefinition {',' ParameterDefinition} ')';
ParameterDefinition = Parameter | ParameterList;
Parameter = '?' Identifier;
ParameterList = '??' Identifier;
\end{lstlisting}

You can use a single parameter to parameterize the following elements of a filter module:%
\begin{itemize}[noitemsep]
\item Internal type
\item External type
\item External initialization
\item Condition declaration
\item Filter type
\item Filter arguments
\item Target
\item Selector
\end{itemize}

Lists of parameters can only be used in the selector of a matching.

\section{Semantics}
The parameters are variables that can be used in the scope of a filter module. 
The values of the parameters are assigned with the filter module binding.
Parameter types are inferred automatically; the declaration of a parameter is only the identifier of the parameter with a prefix.

\section{Examples}
The first example, \autoref{lst::ARM:fmp:example1}, shows a filter module that handles logging. 
Logging is a common concern and it is likely that a filter module like this will end up in a concern library. 
The given arguments, in line 1 of the example, are an external declaration, which can be an object or a method that will return an object, a method reference, which must refer to a method that returns a boolean value, and a set of selectors.
The external and condition declaration are like an ordinary declaration only the right hand side of the declaration has now a parameter, thus this part is known when the filter module is bound to an object.
In the filter \lstinline!??setOfSelectors! is used as argument for the selector part, it means that every item of the list will be evaluated, in this example this means that there will be a name matching for the functions \lstinline!print! and \lstinline!printAll!, which are given in the filter module binding in line 12.
In that binding the other two arguments are given as well.

\begin{lstlisting}[caption={Example of a filter module for logging},label=lst::ARM:fmp:example1,
style=listing,language=ComposeStar,float=tpb]
filtermodule logging(?externalDeclaration, ?methodreference, ??setOfSelectors){
  externals
    logger : Logger = ?externalDeclaration;
  conditions
    c : ?methodreference;
  inputfilters
    m : Meta = {c => [*.??setOfSelectors] logger.log}
} 

superimposition
  filtermodules
    self <- logging(Logger.instance(), inner.isActive(), {print, printAll});
\end{lstlisting}

The second example, \autoref{lst::ARM:fmp:example2}, is a generic single inheritance filter module.
The internal declaration is like an internal declaration with a parameter and with the filter module binding a proper class type must be provided. 
The filter in line 5 first uses a signature matching on inner to have the possibility of overriding methods.
 \autoref{lst::ARM:fmp:example3} demonstrates a generic encryption filter module with a parameterized filter type and a parameter list for the selector of the matching part.

\begin{lstlisting}[caption={Example of a generic inheritance filter module}, label = lst::ARM:fmp:example2,
style = listing, language=ComposeStar, float = tpb]
filtermodule inherit(?internaltype){
  internals
    parent : ?internalType;
  inputfilters
    d : Dispatch = {<inner.*> inner.*, <parent.*> *.*}
} 
\end{lstlisting}

\begin{lstlisting}[caption = {Example of an encryption filter}, label = lst::ARM:fmp:example3,
style = listing, language =ComposeStar, float = tpb]
filtermodule encryption(??setOfSelectors, ?filtertype){
  outputfilters
    e : ?filtertype = {[*.??setOfSelectors]}
} 
\end{lstlisting}

\section{Legality rules}
\begin{itemize}[noitemsep]
\item The identifiers of the parameters must be unique for one filter module, so declaring both \lstinline!?para!
and \lstinline!??para! is not allowed. 
So an identifier cannot occur in one filter module with different prefixes;
\item Although you do not need to specify types, there is a typing system for the parameters. 
If there is a typing error the compiler will let you know, for more information on the technical details consult
the implementation details in~\cite{Doornenbal2006};
\item When a filter module defines parameters, the same amount of parameters must be provided in the superimposition.
\item A list of parameters can only be used in the selector of a matching part.
\end{itemize}

\faq{
\para{FAQ}
\question {Because ?parameter is a single parameter and ??parameter a list of parameters, does that mean
that ???parameter is a list of parameter lists?}
\answer{No, because all ``???(?)\mult'' prefixes are too complex to use and during the design of the filter module parameters no general usage could be found for ???parameter.}

\question{How can you determine the types, for example an internal type or a selector, that you need to fill in the filter module binding?}
\answer{You must derive the type information from the filter module specification.}

\para{Hints}
%%michielh: does not belong here?
%The given arguments for the parameters can be a string or a reference to an object.
%Compose* handles the type conversions for string to object reference and vice versa, but when a wrongly typed argument is provided, Compose* will give an error.

In \autoref{lst::ARM:fmp:example1}, line 3, the type of the external is fixed and the initialization expression is made generic with a parameter. 
This construction is more robust than the construction where the type of the external is a parameter as well.
Because with a static type we know that initialization expression must give back the static type, or a type that inherits from the static type.
So if the external declaration is type sound then we also know a part of the signature. 
If we take \autoref{lst::ARM:fmp:example1}, we see that we can write a reusable logging filter module.
The user that reuses this filter module must provide an object with a certain signature.
In this case we want to have at least the method \lstinline !log! in the signature. 
With an external with a generic type it is not possible to enforce this.
}

\comments{
The prefixes are inspired by Sally~\cite{Sally} and LogicAJ~\cite{Logicaj}, which both extend AspectJ with parameters. 
They make the distinction between a single parameter and a parameter list with the different prefixes ``?'' and ``??''. 
When we were looking to parameters we wanted a way to make them distinct from literals, so adding a prefix is
an obvious solution for this.

\para{Typing}
Typing is left out of the grammar because we infer types automatically in Compose*.
This means that the user does not have to define the type, which lowers the complexity of the syntax. 
If there is a problem with incompatible types or parameters used on two places and those places do require
different types, then \Compose* will give an error. 

\para{Not implemented warning}
Filter module parameters are currently only implemented for selectors.
An error will be given when parameters are used in any of the other specified locations.
}

%\pending{}

\furtherreading{
The filter module parameters are introduced in \cite{Doornenbal2006}, here you can find more information on how the syntax was formed and what the other alternatives were.
}