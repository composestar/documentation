% ARM entry for selector.
\chapter{Selectors} \label{chapter:selector}
The selector field is where you can make a selection of certain program elements. The selections
can be used to bind filter modules or annotations, and it is possible to use the selection as an argument for a filter module parameter.

\section{Syntax} %\label{insert_label}
\begin{lstlisting}[caption = {Selector syntax}, label = lst::ARM:sel:syntax, style = listing, language = ebnf, float = tpb]
Superimposition = '{' [Conditionals] [Selectors] [FilterModuleBindings] [AnnotationBindings] [Constraints] '}';
Selectors = 'selectors' { Identifier '=' '{' ( LegacyExpr | PredicateExpr) '}' ';' };
LegacyExpr = '*' ( ':' | '=' ) Fqn;
PredicateExpr = Identifier '|' ? prolog expression ?;
\end{lstlisting}

The identifier of a selector must be unique for the superimposition where you declare it in. 
There are two available selector expressions: the legacy expression (usage discouraged) and the predicate expression.

The legacy expression always starts with a \lstinline|*| followed by an operator and the fully qualified name of a class.
The \lstinline|=| selects the specified class, and \lstinline|:| selects the specified class and all its subclasses.

The predicate selector consists of two parts, the first part is the identifier that will contain the result of the evaluated predicate expression. 
The second part contains the predicate expression, which is a prolog expression.
This expression should create a result with the identifier specified in the first part.
The list of possible predicate expressions can be found in \autoref{appendix:selectorfunctions}.

\section{Semantics}
With the declaration of a selector you make a selection of program elements and you label these with an identifier.
This identifier can be used in the different bindings and as filter module argument. 
There is a default selector ``self'' which selects a class with the same name as the concern.

\section{Examples}
\begin{lstlisting}[caption={Some selector examples}, label = lst::ARM:sel:example1, style=listing, language =ComposeStar, float = tpb]
selectors
  strategy = 
    { Random | isClassWithName(Random, 'pacman.Strategies.RandomStrategy') };
  levels = 
    { C | isClassWithNameInList (C, ['pacman.World', 'pacman.Pacman']) };
  player = 
    { C | methodHasAnnotationWithName (M, 'Jukebox.StateChange'), classHasMethod(C, M)};
\end{lstlisting}
In \autoref{lst::ARM:sel:example1} there are three example selections demonstrated. 
The selectors are written with queries, that are modeled to a common language model.

\section{Legality rules}
\begin{itemize}[noitemsep]
\item A selector identifier must be unique for the superimposition block it is declared in;
\item A selector identifier may not be named self.
\end{itemize}

%\faq{
%\para{FAQ}
%\para{Hints} -> no hints
%}
\comments{The design issues for the new selectors are described in~\cite{Havinga2005}, however the usage changes with the addition of new features and some things can be added to the selector language.

\para{Extended Selector Syntax}
\begin{lstlisting}[caption={A selector with extended syntax}, label = lst::ARM:sel:comment1,
style=listing, language =ComposeStar, float = tpb]
selectors
  selection(C, M, A) = {MethodHasAnnotation(M, A),
           classHasMethods(C, M), classWithName(C, "certainName")};
  selectionB(C) = selection(C, M, A);
filtermodules
  selection.C <- fm(selection.A);
\end{lstlisting}
With the introduction of filter module parameters, we also want to select program elements to use as arguments in the
filter module binding. And it would be even better if we can get multiple selectors from one query. So the new proposed
layout is like \autoref{lst::ARM:sel:comment1}, were it is possible to use the variables of the query further on in
the superimposition. Another proposal is to reuse queries in other queries, which offer good reusable code.
 }
%\dotnetcomment{}
%\javacomment{}
%\ccomment{}
%\pending{}
\furtherreading{
In \cite{Havinga2005} the new selector language is introduced. 
The considerations on the chosen design and the implementation can be found there.
}