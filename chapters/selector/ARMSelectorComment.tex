The design issues for the new selectors are described in~\cite{Havinga2005}, however the usage changes with the addition of new features and some things can be added to the selector language.

\para{Extended Selector Syntax}
\begin{lstlisting}[caption={A selector with extended syntax}, label = lst::ARM:sel:comment1,
style=listing, language =ComposeStar, float = tpb]
selectors
  selection(C, M, A) = {MethodHasAnnotation(M, A),
           classHasMethods(C, M), classWithName(C, "certainName")};
  selectionB(C) = selection(C, M, A);
filtermodules
  selection.C <- fm(selection.A);
\end{lstlisting}
With the introduction of filter module parameters, we also want to select program elements to use as arguments in the
filter module binding. And it would be even better if we can get multiple selectors from one query. So the new proposed
layout is like \autoref{lst::ARM:sel:comment1}, were it is possible to use the variables of the query further on in
the superimposition. Another proposal is to reuse queries in other queries, which offer good reusable code.
