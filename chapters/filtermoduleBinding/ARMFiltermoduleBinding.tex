\chapter{Filter Module and Annotation Binding} \label{ch::arm::fmb}
The binding parts are the places to bind certain properties to elements of a selector. Currently we can bind filter modules
and annotations to classes.

\section*{Syntax}
\begin{lstlisting}[caption = {Bindings syntax}, label = lst::ARM:fmb:syntax,
style = listing, language = ebnf, float = tpb]
FilterModuleBindings ::= `filtermodules' (FilterModuleBinding `;')*
FilterModuleBinding ::= Identifier WeaveOperator 
                        (FilterModuleReference-LIST 
                        | `{' FilterModuleReference-LIST `}')
FilterModuleReference ::= Identifier [`(' Argument-LIST ')']
AnnotationBindings ::= `annotations' (AnnotationBinding `;')*
AnnotationBinding ::= Identifier WeaveOperator 
                        (Identifier-LIST | `{' Identifier-LIST `}')
WeaveOperator ::= `<-'
\end{lstlisting}
The syntax in \autoref{lst::ARM:fmb:syntax} shows the common layout for bindings. The weave operator is legacy, from
the times when unweaving was still considered. The left hand side identifier must be a selector reference and the
right hand side depends on what kind of binding you are doing.

\section*{Semantics}
The meanings of the bindings is that you superimpose whatever is on the right hand side on the left hand side.
Currently \Compose* only binds filter modules and annotations. 
The bindings gets executed during the initialization of the application.

\section*{Examples}
\begin{lstlisting}[caption={Examples of bindings}, label = lst::ARM:fmb:example1,
style=listing, language =ComposeStar, float = tpb]
filtermodules
  selA <- filtermoduleA;
  selB <- filtermoduleB(``Argument'', selC), filtermoduleA;
annotations
  selA <- anAnnotation;
\end{lstlisting}
A binding has a left hand side and a right hand side, the left hand side contains a selector identifier,
which contains a selection of classes. The right hand side has one or more filter module identifiers with the appropriate
arguments or one of more annotations. Some examples are shown in \autoref{lst::ARM:fmb:example1}.

\section*{Legality Rules}
\begin{itemize}[noitemsep]
\item The used selectors must be defined in the same superimposition block;
\item The used filter module references must point to an existing filter module.
\end{itemize}

%faq{} -> geen vragen, moet alemaal duidelijk zijn
\comments{
\para{More Binding Operations} % aka puinruimen
In the begin days of \Compose* there were ideas on binding and unbinding on events, however these never got implemented.
This leaves us with the only weave operator: ``\lstinline|<-|''.

\para{Introduction of Parameters}
The binding of filter modules differs from the annotation binding because it has the possibility to attach arguments.
In this argument list we allow constructors to get an unique external that is only shared for that binding.
}
\dotnetcomment{
\para{Introduction of Parameters}
Both constructs mentioned in the comments do not work currently, only literals can be used as arguments for now.
}
%\javacomment{}
%\ccomment{}
%\pending{}
\furtherreading{
More on filter module binding can be found in~\cite{salinas:ms01}. Annotation binding is introduced in~\cite{Havinga2005}.
}