% empty template for an ARM chapter.
\chapter{Superimposition Conditions}\label{cha:Superimposition_Conditions}

The superimposition conditions are much like the conditions defined in the filter modules.
Except that they can be used for conditional superimposition of filter modules.
This will allow the binding of filter modules to be depended on a runtime variable.

\section{Syntax}

\begin{lstlisting}[caption={Superimposition Conditions Syntax}, label=lst::ARM:sicond:syntax, style=listing, language=ebnf, float=tpb]
Superimposition = '{' [Conditionals] [Selectors] [FilterModuleBindings] [AnnotationBindings] [Constraints] '}';
Conditionals = 'conditions' {SiCondition ';'};
SiCondition = Identifier ':' Fqn ['(' Arguments ')']
\end{lstlisting}

Just like the conditions defined in a filter module, these conditions consist of a identifier and a method that returns a boolean value.
The condition identifier must be unique within the superimposition block.

\section{Semantics}

Each condition is bound to a method.
The method must be a reference to a static method in a defined class.
The condition can be used in the filter module binding part of the superimposition.
It is not possible to use selector values as class reference.
Selectors are often a set of classes, which makes it unclear which class to use for the method binding.

\section{Examples}

\begin{lstlisting}[caption={Some condition examples}, label = lst::ARM:sicond:example1, style=listing, language =ComposeStar, float = tpb]
conditions
      loggingEnabled : BasicTests.ConditionalSuperImpositionTests.LoggingEnabled;
		  timingEnabled : BasicTests.ConditionalSuperImpositionTests.TimingEnabled;
\end{lstlisting}

\autoref{lst::ARM:sicond:example1} shows two defined conditionals. They both point to a static method in the class \lstinline|BasicTests.ConditionalSuperImpositionTests|.

\section{Legality Rules}
\begin{itemize}[noitemsep]
\item The identifier must be unique within the superimposition block;
\item The method must be a public static method;
\item The static method must return a boolean value.
\end{itemize}

%\faq{}

\comments{
\para{Method arguments}
The method arguments are not supported and implemented.
It is currently unclear what values could be passed to these static methods.
It might be a option to use the values of the concern parameters as explained in \autoref{chapter:concern}.

\para{Port support}
Superimposition conditionals are currently only supported in StarLight. 
The runtime ports (\Compose*[DotNet] and \Compose*[Java]) do not support the usage of conditionals.
They will accept it in the language, but they are disregarded in the execution.
}
%\dotnetcomment{}
%\javacomment{}
%\ccomment{}

%\pending{}

%\furtherreading{}
