% empty template for ARM entry. By Dirk Doornenbal. V1.5 2 mei 2006
\chapter{Filter Substitution Part}
The substitution part of a filter is where you can specify the changes in a message. Just like the
matching part it consists of a target and a selector.

\section*{Syntax}
The syntax is almost similar to the matching part and is shown is \autoref{lst::ARM:sp:syntax}. The difference is
that the substitution part does not have any brackets, creating a visible difference between matching and substitution part. Another difference is that it is not possible to have a list of substitution parts.

\begin{lstlisting}[caption={Filter substitution part syntax},label=lst::ARM:sp:syntax,style = listing,language = ebnf,float=tpb]
FilterElement = [ConditionExpression ConditionOperator] MessagePattern
MessagePattern ::= Matching [SubstitutionPart]
                   | MatchPattern
                   | `{' Matching (`,' Matching)* `}' [SubstitutionPart] 
Matching ::= SignatureMatching | NameMatchingSubstitutionPart = [Target `.'] Selector
\end{lstlisting}

\section*{Semantics}
A substitution part gets executed when a filter matches, how it gets executed depends on the filter type.
The meaning of the substitution part is simple, if there is a wild card on the target or selector spot the attribute of the message
stays the same. Otherwise the attribute gets substituted with the given value.

\section*{Examples}
In \autoref{lst::ARM:ARMSP:example1} three out of the four possible substitution part combinations are demonstrated. The matching parts are different as well to get a filter specification that means something.
The fourth is that the target stays the same and that the selector changes.

\begin{lstlisting}[caption={Some possible substitution parts},label=lst::ARM:ARMSP:example1,
style=listing,language =ComposeStar,float=tpb]
filtermodule example{
  inputfilters
    d : Dispatch = {[tar.sel] tar1.*, [*.sel] tar2.sel2, <tar.*> *.*}
\end{lstlisting}

\section*{Legality rules}
\begin{itemize}[noitemsep]
\item The used target must be a declared internal, external, the keyword ``inner'', or a single parameter;
\item A selector can be a method or a single parameter;
\item A selector cannot be a list of parameters.
\end{itemize}

\faq{
\para{FAQ}
\question{Why is it not possible to use a parameter list in the selector of a substitution part, like in the matching part?}
\answer{A message can only be send to one method, so a list of methods would be meaningless.}

\para{Hints}
Keep an eye on where you send your messages to. If the combination of target and selector does not exist you get an error,
but it is easier to find the error before it occurs.
}

\comments{
Just as with the matching part we can only work with the target and the selector. With the proposed
new filter layout (\cite{Doornenbal2006}) we can alter more attributes of a message.
}
%\dotnetcomment{}
%\javacomment{}
%\ccomment{}
%\pending{}
%\furtherreading{}