% empty template for an ARM chapter.
\chapter{Constrains}\label{cha::fmconstaints}

With the constraints defined in the super imposition block it is possible to define filter module ordering constraints.
\Compose* will try to find the most suitable filter module ordering for each concern.
When there are no defined constraints this order is arbitrary.
In some cases certain filter modules must be superimposed before others to prevent behavioral conflicts.
The constraint block in the superimposition block allows you to define such constraints.

\section{Syntax}

\begin{lstlisting}[caption={Filter Module Constrains Grammar}, label=lst::ARM:fmconstraint:syntax, style=listing, language=ebnf, float=tpb]
Superimposition = '{' [Conditionals] [Selectors] [FilterModuleBindings] [AnnotationBindings] [Constraints] '}';
Constraints = 'constaints' {Constraint ';'}
Constraint = Identifier '(' FilterModuleRef ',' FilterModuleRef ')';
FilterModuleRef = [Fqn '::'] Identifier;
\end{lstlisting}

Each constraint consists of a constraint type and two filter modules it defines the constraint over.
Currently there is only one constraint type: \emph{pre}.

\section{Semantics}

The constraint definition define a ordering relation between the left filter module and right filter module.
The \emph{pre} constraint defines that the left filter module must come before the right filter module.
With this constraint it is allowed that other filter modules are between the left and right filter module.
The \emph{pre} constraint is an alias for the \emph{soft-pre}.
An other constraint is the \emph{hard-pre} constraint which is just like the \emph{soft-pre} constraint except that it does not allow other filter modules between the left and right filter module.

The filter modules references in the constraints can be either local filter modules or filter modules in other concerns.
The constraints between filter modules are global, they affect all filter module bindings in all superimposition blocks.

\section{Examples}


\begin{lstlisting}[caption={Filter module ordering constraints}, label = lst::ARM:fmconstraints:example1, style=listing, language =ComposeStar, float = tpb]
superimposition{
  constraints
    pre(FM1, FM2);
    pre(FM1, Examples.Otherconcern::Foo);
    pre(Examples.Otherconcern::Foo, Examples.Otherconcern::Bar);
}
\end{lstlisting}

The example in \autoref{lst::ARM:fmconstraints:example1} shows thre different constraints.
The first declares a constraint between two locally declared filter modules.
The second constraint is between a local and external filter module.
And the last one declares a constraint between external filter modules.
When declared constraints for external filter modules you should watch out not to create conflicting constraints.

\section{Legality Rules}

\begin{itemize}[noitemsep]
\item Local filter modules must exist, external filter modules will be ignored when they do not exist;
\item Constraints must not result in a cycle, e.g. \lstinline|pre(FM1, FM2);pre(FM2, FM1);| is illegal.
\end{itemize}

\faq{
\para{Hints}
You can use concern with only a constraint section to declare the constraints of all filter modules used in the whole program.
This is especially useful when generic filter filter module only concerns are used.
}

%\comments{}
%\dotnetcomment{}
%\javacomment{}
%\ccomment{}

%\pending{}

\furtherreading{
More information about the constraints can be found in \cite{Nagy2006} and \cite{Vinkes2004}.
Information about conflicts created by incorrect filter module ordering are described in \cite{Durr2004}.
}
