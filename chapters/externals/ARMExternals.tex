% empty template for ARM entry. By Dirk Doornenbal. V1.5 2 mei 2006
\nomenclature{fqn}{Fully Qualified Name}
\chapter{Externals}
Externals are objects that are instantiated outside a filter module.
They can be instances which are used in multiple places in the application, for instance
as a Logger object, or they are shared between multiple filter module instances.

\section*{Syntax}
\begin{lstlisting}[caption = {Externals syntax}, label = lst::ARM:ext:syntax,
style = listing, language = ebnf, float = tpb]
FilterModule ::= `filtermodule' FilterModuleName [FilterModuleParameters] `{'
                  [Internals] [Externals] [Conditions]
                  [InputFilters] [OutputFilters] `}'
Externals ::= `externals' (Identifier `:' Type 
              `=' InitialisationExpression `(' [ ArgumentList ] `)' `;')* 
InitialisationExpression ::= MethodReference                
\end{lstlisting}
As we can see in \autoref{lst::ARM:ext:syntax}, we can split up the external in three parts: the identifiers, type, and the initialization. The parts are
separated with syntactic sugar, between the identifiers and type there is a colon (`:') and
between the type and initialization an equation mark (`=').
The initialization expression is the method that returns the reference to the external object.
It is possible to use arguments in the initialization expression.

\section*{Semantics}
\begin{figure}[tpb]
	\centering
	\includegraphics[style=thirdheight]{ARM-externalInitialization}
	\caption{Schema of \autoref{lst::arm::ext:example1}}
	\label{fig::arm::ext:schema2}
\end{figure}
If we take \autoref{lst::arm::ext:example1} and look how the instances are created, we get \autoref{fig::arm::ext:schema2}.
Every instance of the filter module \lstinline!dynamicscoring! gets a reference to the common object Score.
If we draw a schema of \autoref{lst::ARM:ext:example2} then we get a different picture, because every instance of delegatePlanning gets a pointer to one of the two shared externals.

Getting a shared object can be done on several ways, however you must write a construct that holds the objects so that
you can select them, thus you should keep a collection of externals. If there is only one instance of an external, then
the Singleton pattern is a good pattern to apply.

\section*{Examples}
\begin{lstlisting}[caption = {Dynamic scoring filter module from the Pacman example}, label = lst::arm::ext:example1,
style = listing, language = ComposeStar, float=tpb]
filtermodule dynamicscoring{
  externals
    score : pacman.Score = pacman.Score.instance();
  inputfilters 
    score_filter  : Meta = { [*.eatFood] score.eatFood, [*.eatGhost] score.eatGhost, 
			[*.eatVitamin] score.eatVitamin, [*.gameInit] score.initScore,
			[*.setForeground] score.setupLabel }
}

superimposition{
  selectors
		scoring = { C | isClassWithNameInList
		            (C, ['pacman.World', 'pacman.Game', 'pacman.Main']) };
  filtermodules
    scoring <- dynamicscoring;
}
\end{lstlisting}
\begin{lstlisting}[caption = {An external without the singleton construction}, label = lst::ARM:ext:example2,
style = listing, language = ComposeStar, float = tpb]
filtermodule delegatePlanning(?external){
  externals
    e : Secretary = ?external;
  inputfilters
    d : Dispatch = {<inner.*> *.*, <e.*> e.*};
}

superimposition{
  ...
  filtermodules
    selA <- delegatePlanning(Secretary());
    selB <- delegatePlanning(Secretary());
}
\end{lstlisting}
In the Pacman example, the filter module \lstinline!dynamicscoring! uses an external
to keep track of the score (\autoref{lst::arm::ext:example1}). For this concern it means that
every instance of the classes \lstinline!pacman.World!, \lstinline!pacman.Game!,
and \lstinline!pacman.Main! gets its own
instance of the filter module dynamicscoring and each of the instances of dynamicscoring has
a reference to the same instance of \lstinline!pacman.Score!. The construction with scoring in Pacman is
known as the Singleton pattern \cite{Gamma95}. It is also possible to get constructions without the Singleton pattern as demonstrated in \autoref{lst::ARM:ext:example2}, where an object is used as parameter and is assigned through the filter module binding.

\section*{Legality rules}
\begin{itemize}[noitemsep]
\item The identifier of an external must be unique for the filter module where it is declared in;
\item The type must be a fully qualified name of a class and this class must exist;
\item The initialization expression must point to a valid method, this often means that you use a static method.
\end{itemize}

\faq{\para{FAQ}
\question{How do I create an external that is unique for a certain number of filter module instances?}
\answer{If you want to use an external that is not shared with every instance of the filter module, you can use the filter module parameters to define the different instances for the external.}}
\comments{The advantage of using the Singleton pattern becomes visible when we would use a
parameter as initialization String. In the given example \autoref{lst::arm::ext:example1},
we see that we can address the same instance of Score from three filter modules and that there will only be one
instance of Score in the application. We could do the same without a filter module, by just
adding the Score object to every class. The advantage of the filter is that when we would write
\lstinline[breaklines=false]!score : pacman.Score = ?instance;!, then we only need to change the initialization String
in one spot instead of all the classes where we want to access Score.

\para{The Alternative to Singletons}
With the addition of filter module parameters it is also possible to use a given object from the parameters as an
external. This fixes the old limitation that you had to choose between a singleton construction, which results in that
every filter module points the same object, or to use an internal, which results in that very instance has it own instance of the
internal. In \autoref{lst::ARM:ext:example2} we have an example were two selections, selA and selB, get a different 
instance of the class secretary. Using an object as a filter module parameter makes it possible to use an instance for
a certain selection of filter module instances instead of all filter module instances.

\para{Arguments in the Method Calls}
Currently it is only possible to use the Singleton design pattern for an external declaration. 
This construct is mentioned earlier in~\cite{Doornenbal2006}
for arguments in conditions calls.

\para{Introducing Static Methods}
In the old syntax it was possible to declare an external without an initialization expression. So the syntax was:
\begin{lstlisting}[style=listingWithNoNumbers,language=ebnf]
Externals ::= `externals' (Identifier `:' Type 
              [`=' InitialisationExpression `(' [ ArgumentList ] `)'] `;')*            
\end{lstlisting}

The purpose of such construct is that with an external declaration without an initialization expression it is possible to introduce static methods. A static method from a class can be used without having instantiated an object from this class.
Because the target needs to be an internal, external, or the keyword ``inner'', the only way you are able to substitute the target, so that it is send to a static method, is by declaring the class as an external.
However, this construct has been removed because the construct probably was not placed in the syntax to support the usage of static methods and that it is just an error in the grammar. The theory mentioned above is a possible explanation what the old syntax could mean.}
\dotnetcomment{\para{Arguments in the initialization expression}
The possibility to use arguments in the initialization expression is currently not supported
by Compose*.Net. This is because the grammar does not support it and it is not certain whether
the other parts can handle them. So for now you should try to work around it, it will be
checked whether it can be fixed as soon as the filter module parameters are being implemented.
}
%\javacomment{}
%\ccomment{}
\pending{Which are declared first, the internals or the externals? 
To be more concrete, will \autoref{lst::ARM:fmp:pending1} work or not? 
And does it always work? 
If the internals are earlier declared then the external, then the example works and there are no further problems. 
(The given example is probably not the best way to combine internals and externals.)

\begin{lstlisting}[caption = {A filtermodule with an external from an internal},label=lst::ARM:fmp:pending1,style=listing,language=ComposeStar,float=tpb]
filtermodule pending(){
  internals
    i : Distribution;
  externals
    e : Object = i.getSingleton();
}
\end{lstlisting}}
%\furtherreading{}