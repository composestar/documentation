% ARM entry for FIlter Lement, aka the new filter syntax
%
% details on the used operators notebook 01-03-06, husselen vn argumenten in een method
\chapter{Filter Elements} \label{ch::arm::fe}

\section{Examples}%\label{insert_label}
% one or more examples

\section{Syntax} %\label{insert_label}
% hoe does it look like, what can you write down

\section{Semantics} %\label{insert_label}
% what does the constructions mean?

\section{Legality Rules}

\faq{}
\comments{
% \para{Removed from the Old Syntax}
% removal of ~>

%\examples en commentaar in H:\codevoorbeelden\filterparametersV1\

% in het h\gesprekken\voornbereiding wordt het restaurant voorbeeld aangehaald, wel wat veroudert, echter het vegeatarier oplossing is wel geniaal, en wordt daarom ook niet toegepast.

%ARGH type anlomaiels en het koppelen van filters wordt pas interssant als je naar de nieuwe syntax gaat.
% like type anomalies -. notebook 16-01-06 -> gaat uit van de nieuwe syntax
% ook in H:\codevoorbeelden\filterobjectpartV1\example.....doc staat een punt waarom je niet verder hiermee wilt, het komt vooral door het gebrek aan ( )
%
% ook een keuke over 3 filters die exact hetzelfde zijn.......
%
%\subpara{Filter labelling} -> dmv van aparte filter deifnties en een combinatie stuk zou je dit probleem kunnen fixen, staat beschreven in: H:\codevoorbeelden\filterobjectpartV1\example.....doc
% extensions for filter, also filter inheritance 

}
%\dotnetcomment{}
%\javacomment{}
%\ccomment{}
%\pending{}
%\furtherreading{}