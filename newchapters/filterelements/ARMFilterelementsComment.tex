\para{Removed from the Old Syntax}
Several options of the old syntax have not been ported to the new syntax. Two operators which are now
removed are the \lstinline|=>| and \lstinline[language=ebnf]|~>|, this is done because they have become
we can use logical operators on the matching elements.

\para{Separated Substitution Elements}
In the new design we can see when the first substitution element is being parsed, however it is not certain
this still works when we also allow functions on message attributes. Because these substitution elements can also been seen as
matching elements. In that scenario we must introduce a separator symbol like ``\lstinline|/|'', to mark the
transition between condition and matching elements and the substitution elements. This separator comes
from the original ideas, but is dropped because we could parse the filter correctly without. When we would
come to the conclusion that this is indeed impossible, we need to add it again to the syntax.

\para{Filter Operators}
\begin{lstlisting}[caption = {Dynamic strategy filter module in Pacman, with filter operators}, label=lst::arm::se:comment1,style = listing, language = ComposeStar,float=[tpb]]
inputfilters
prefilter : Check ={selector = getNextMove} AND
(
  stalker_filter : Dispatch = {!pacmanIsEvil, target := stalk_strategy } OR
  flee_filter : Dispatch = {target := flee_strategy }
}
\end{lstlisting}
\begin{lstlisting}[caption = {Dynamic strategy filter module in Pacman, done in one filter}, label=lst::arm::se:comment2,style = listing, language = ComposeStar,float=[tpb]]
inputfilters
behavior : Dispatch ={selector = getNextMove AND ((!pacmanIsEvil, 
                      target := stalk_strategy ) OR target := flee_strategy )
\end{lstlisting}
With the introduction of the new filter syntax, we can look again to the filter operators. With filter operators we can rewrite the input filter set of \autoref{lst::arm::se:example1} to \autoref{lst::arm::se:comment1}. However we get again stuck on the parenthesis and this example could be better
solved by allowing filters like \autoref{lst::arm::se:comment2}., the only drawback is that such filter
constructions are not allowed with the proposed syntax. We should not be eager to add this construct to
the syntax, because the behavior filter is harder to read then the two filter version.