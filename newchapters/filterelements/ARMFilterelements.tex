% ARM entry for Filter Element, aka the new filter syntax
\chapter{Filter Elements} \label{chapter:newfilterelement}
The filter elements are the parts of the right hand side of a filter. They are the proposed
new filter syntax, as described in~\cite{Doornenbal2006}.
The layout for the condition, matching, and substitution part, are alike in this syntax and therefore they
have been combined into one chapter.

In the filter element part it is possible to specify the matching and substitution behavior of a filter.
The old syntax is still usable, because it can be transformed to the new syntax. How this transformation is
done is shown in~\cite{Doornenbal2006}.

\section*{Syntax}
\begin{lstlisting}[caption = {Filter element syntax},label=lst::arm:fe:syntax,style=listing,language=ebnf, float=tpb]
GeneralFilterSet ::= GeneralFilter (FilterCompositionOperator GeneralFilter)*
FilterCompositionOperator ::= `;'
GeneralFilter ::= FilterName `:' FilterType [`(' ActualFilterParameters `)'];
                         `=' `{' [FilterElements] `}'
FilterElements ::= FilterElement (ElemCompositionOperator FilterElement)*
ElemCompositionOperator ::= `;'

FilterElement ::=  [ConditionAndMatchingPart `,'] AssignmentPart 
                   | ConditionAndMatchingPart

ConditionAndMatchingPart ::= OrExpr
OrExpr ::= AndExpr (`OR' AndExpr)*
AndExpr ::= PrimaryExpr (`AND' PrimaryExpr)*
PrimaryExpr ::= [NOT] PlainExpr;
PlainExpr ::= `(' OrExpr `)' | ConditionPart | MatchingPart

ConditionPart ::= ConditionLiteral
ConditionLiteral ::= `True' | `False' | ConditionName
MatchingPart ::= FieldName MatchingOperator (Value | `{' Value-LIST `}')
                 | FieldOperation
MatchingOperator ::= equals | equalsOne | equalsAll | sig | OtherOperators

AssignmentPart ::= Assignment (AND Assignment)*
Assignment ::= FieldName `:=' Value | `{' Value-LIST `}' | FieldOperation
AssignmentOperator ::= `:=' | OtherOperators

FieldName ::= Identifier
FieldOperation ::= fqnwithArguments
AND ::= `&' | `,'
OR ::= `|'
NOT ::= `!'
Value ::= Identifier | Number | Parameter | ParameterList
Methodcall ::= Identifier `(' Value-LIST `)'
OtherOperator ::= (* To Be Specified *)
\end{lstlisting}
\autoref{lst::arm:fe:syntax} shows the syntax for the filter elements. The conditions and matching elements
are first, followed by the substitution elements. The conditions elements are the labels defined in the condition field of the filter module. The matching elements are the name of a message property, an operator, and one or more values. It is also possible to use a function of the message.
The set of usable operators in the matching has not been fully worked out yet.
It is possible to combine both with the standard set of logical operators, AND, OR, and NOT.
The substitution element has only the ``:='' operator to assign a value to a message property.

\section*{Semantics}
The condition and matching elements provide boolean values which can be used in a logical expression to determine whether a filter matches or not. If a filter matches the substitution elements are being
executed.

The matching elements can have predefined operators, the proposed ones are ``='' and ``sig''.
The operator ``='' is
short hand for the equals operator and for a set it means of the out of the set. And the operator ``sig'' means
signature matching, this operator can only be used to check on a selector whether it is in a certain signature.
For lists we also have the ``equalsOne'' and ``equalsAll''. Other operators have to be assigned yet.
The matching elements has only the assignment operator, which is defined as ``:=''.

The comma's between the elements are AND operators. It also has become possible to write filters without any condition or matching elements, so that the substitution elements are always executed.

\section*{Examples}
\begin{lstlisting}[caption = {Dynamic strategy filter module in Pacman in new filter syntax}, label=lst::arm::se:example1,style = listing, language = ComposeStar,float=[tpb]]
filtermodule dynamicstrategy{
  internals
    stalk_strategy : pacman.Strategies.StalkerStrategy;
    flee_strategy : pacman.Strategies.FleeStrategy;
  conditions
    pacmanIsEvil : pacman.Pacman.isEvil();
  inputfilters
    stalker_filter : Dispatch = {!pacmanIsEvil, selector = getNextMove, 
                                 target := stalk_strategy };
    flee_filter : Dispatch = {selector = getNextMove, target := flee_strategy }
}
\end{lstlisting}
\begin{lstlisting}[caption = {Custom logging filter module, in new filter syntax}, label=lst::arm::se:example2,style = listing, language = ComposeStar,float=[tpb]]
filtermodule logger(??inputMethods, ??outputMethods){
  externals
    logger : Logger = Logger.instance();
  conditions
    isLoggingOn : logger.isOn();
  inputfilters
    inlog : Meta = {isLoggingOn, selector = ??inputMethods, target := logger,
                    selector := log}
  outputfilters
   outlog : Meta = {isLoggingOn, selector = ??outputMethods, target := logger,
                    selector := log}
}
\end{lstlisting}
\begin{lstlisting}[caption={The mass mailer from the filter syntax analysis},label= lst::arm::se:example3,style=listing,language=Composestar,float=[tpb]]
filtermodule authorization(?rules){
internals
  ruleBase = ?rules; 
externals
  mailer = MassMailer.instance();
conditions
  hasAccesLevelThree : ruleBase.hasAccesLevelThree;
  hasAccesLevelFive : ruleBase.hasAccesLevelFive;
inputfilters
  rule1 : Dispatch = {hasAccesLevelThree, sender = admin, selector sig mailer,
                      target := mailer};
  rule2 : Dispatch = {hasAccesLevelFive, sender = secretary, selector = mailer,
                      target := mailer};
  rule3 : Dispatch = {!(timestamp > 12.00 & timestamp < 15.00),
    (sender = (secretary | admin))| hasAccesLevelThree,
     selector sig mailer, target := mailer};
  rule4 : Dispatch = {!(timestamp > 8.00 & timestamp < 16.00) selector sig mailer,
                      target := mailer};
  error : Error = {}  
}
\end{lstlisting}
In the old syntax entry in the reference manual we have seen two examples, \autoref{lst::arm::fil:example1} and \autoref{lst::arm::fil:example2}, these 
can be rewritten to the new syntax. The result for Pacman is demonstrated in \autoref{lst::arm::se:example1}
and the other is demonstrated in \autoref{lst::arm::se:example2}. \autoref{lst::arm::se:example3} is
an example from~\cite{Doornenbal2006}.

\section*{Legality Rules}
\begin{itemize}[noitemsep]
\item It is not possible to place an assignment part before a condition or matching element;
\item When a attribute of a message is used, it must be available for that type of message.
\end{itemize}

%\faq{} -> no faq and hints
\comments{\para{Removed from the Old Syntax}
Several options of the old syntax have not been ported to the new syntax. Two operators which are now
removed are the \lstinline|=>| and \lstinline[language=ebnf]|~>|, this is done because they have become
we can use logical operators on the matching elements.

\para{Separated Substitution Elements}
In the new design we can see when the first substitution element is being parsed, however it is not certain
this still works when we also allow functions on message attributes. Because these substitution elements can also been seen as
matching elements. In that scenario we must introduce a separator symbol like ``\lstinline|/|'', to mark the
transition between condition and matching elements and the substitution elements. This separator comes
from the original ideas, but is dropped because we could parse the filter correctly without. When we would
come to the conclusion that this is indeed impossible, we need to add it again to the syntax.

\para{Filter Operators}
\begin{lstlisting}[caption = {Dynamic strategy filter module in Pacman, with filter operators}, label=lst::arm::se:comment1,style = listing, language = ComposeStar,float=[tpb]]
inputfilters
prefilter : Check ={selector = getNextMove} AND
(
  stalker_filter : Dispatch = {!pacmanIsEvil, target := stalk_strategy } OR
  flee_filter : Dispatch = {target := flee_strategy }
}
\end{lstlisting}
\begin{lstlisting}[caption = {Dynamic strategy filter module in Pacman, done in one filter}, label=lst::arm::se:comment2,style = listing, language = ComposeStar,float=[tpb]]
inputfilters
behavior : Dispatch ={selector = getNextMove AND ((!pacmanIsEvil, 
                      target := stalk_strategy ) OR target := flee_strategy )
\end{lstlisting}
With the introduction of the new filter syntax, we can look again to the filter operators. With filter operators we can rewrite the input filter set of \autoref{lst::arm::se:example1} to \autoref{lst::arm::se:comment1}. However we get again stuck on the parenthesis and this example could be better
solved by allowing filters like \autoref{lst::arm::se:comment2}., the only drawback is that such filter
constructions are not allowed with the proposed syntax. We should not be eager to add this construct to
the syntax, because the behavior filter is harder to read then the two filter version.}
%\dotnetcomment{}
%\javacomment{}
%\ccomment{}
%\pending{}
\furtherreading{
For the full proposal on the new filter element syntax see ~\cite{Doornenbal2006}.
}